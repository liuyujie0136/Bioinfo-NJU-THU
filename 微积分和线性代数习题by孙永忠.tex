\documentclass[UTF8,a4paper,12pt]{ctexart}
\usepackage{amsmath,amsfonts,amssymb} %数学
\usepackage{fancyhdr} %设置页眉页脚
\usepackage{mathrsfs} %花体 \mathscr{}

\pagestyle{fancy}
\lhead{}
\rhead{}
\cfoot{\thepage}
\renewcommand{\headrulewidth}{0pt}

\begin{document}
\title{\heiti{\Huge{微积分和线性代数习题}}}
\author{南京大学数学系\quad 孙永忠 \\\small{\emph{Kuang Yaming Honors School, Nanjing University}}}
\date{\small{2019年7月24日\space 整理}}
\maketitle

\section{常微分方程}
\subsection{求通解}
1. $y''-xf(x)y'+f(x)y=0$

2. $y''-2y'+y=\frac{e^x}{x}$

\subsection{证明}
1. 齐次方程的任一解都可由一组基础解系线性表出。

2. (1) 若$Y(x)$是一个基解矩阵,$P$是非奇异方阵,则$Z(x)=Y(x)P$也是基解矩阵。

\quad\space (2) 若$Y(x),Z(x)$是两个基解矩阵,则存在非奇异阵$P$,使得$Z(x)=Y(x)P$。

3. 刘维尔公式:$W(x)=W(x_0)e^{\int_{x_0}^x tr(A(x))dt}$

4. 比较原理:设二元函数$f(x,y),g(x,y)$均满足常微分方程基本定理条件,$y=\varphi(x),y=\psi(x)$分别是$\frac{dy}{dx}=f(x,y), y\vert_{x=0}=y_0$与$\frac{dy}{dx}=g(x,y), y\vert_{x=0}=y_0$的解,$x\in[0,h]$。若$f,g$满足$f(x,y)\leqslant g(x,y), \forall x,y$,则$\varphi(x)\leqslant \psi(x), x\in[0,h]$。

\subsection{拉普拉斯变换$\mathscr{L}(f)(s):=\int_0^\infty e^{-st}f(t)dt$}
\subsubsection{求作下列方程的拉普拉斯变换}
1. $f(t)=sin(\omega t)$

2. $f(t)=cos(\omega t)$

\subsubsection{用拉普拉斯变换求解方程}
1. $\frac{d^2y}{dx^2}+y=sin(2x), y(0)=0, y'(0)=1$

2. $\frac{d^2y}{dx^2}+\omega^2y=f(x), y(0)=y_0, y'(0)=y_1$

3. $\begin{cases}
\frac{dx}{dt}+y=-asin(\omega t)\\
\frac{dy}{dt}-x=acos(\omega t) \quad x(0)=1, y(0)=0
\end{cases}$

\subsubsection{证明}
对函数$f,g$,称$f*g(t)=\int_0^t f(t-s)g(s)ds$为卷积,求证:
$$\mathscr{L}(f*g)=\mathscr{L}(f)\mathscr{L}(g)$$

\subsection{零点及边值问题}
1. 设$y''+q(x)y=0$中系数$q(x)$满足$q(x)\leqslant M, x\in[a,+\infty), M>0$为一常数,求证:$y''+q(x)y=0$的任一非零解的相邻零点的距离不小于$\frac{\pi}{\sqrt{M}}$。

2. 求解边值问题:$y''+\lambda y=0, y(0)=0, y'(1)=0$,$\lambda$为大于0的常数。

\section{傅里叶级数}
\subsection{计算}
1. 求出$\sum\limits_{n=1}^\infty \frac{sin(nx)}{n}, \sum\limits_{n=1}^\infty \frac{cos(nx)}{n}$的解析表达式。

2. 对于$f(x)=\frac{1}{2}(\pi-x), x\in[0,2\pi)$作周期延拓后求其\emph{Fourier}级数。

3. 将$f(x)=x^2, x\in[0,\pi)$延拓为以$\pi$为周期的函数,求其\emph{Fourier}级数并讨论收敛性。

4. 求$f(x)=x^3, x\in(-\pi,\pi)$的\emph{Fourier}级数。

5. 将$f(x)=x^2$在下列区间上展开成\emph{Fourier}级数。

(1)[-1,1] \quad (2)[0,$\pi$] \quad (3)[0,1] \quad (4)[0,$2\pi$]

\subsection{证明}
1. 用\emph{Riemann-Lebesgue}引理证明:$\int_0^\infty \frac{sinx}{x}=\frac{\pi}{2}$

2. 利用特征函数$\varphi(x)=
\begin{cases}
1, x\in[a,b]\\
0, x\in[-\pi,a)\bigcup(b,\pi]
\end{cases}$证明傅里叶级数的逐项积分。

\subsection{\emph{Fourier}变换}
1. 计算$f(x)=e^{-\vert x\vert}$的\emph{Fourier}变换。

2. 计算$f(x)=\frac{1}{\sqrt{2\pi\sigma}}e^{-\frac{(x-a)^2}{4\sigma}}$的\emph{Fourier}变换。

\section{线性代数}
\subsection{计算}
1. 三阶实对称阵$A$秩为2,$\lambda=6$为$A$的二重特征值,且$\xi_1=\begin{pmatrix} 1\\0\\1 \end{pmatrix}$,$\xi_2=\begin{pmatrix} 1\\3\\-2 \end{pmatrix}$为$\lambda=6$的特征向量,求$A$。

2. 分别利用正交变换法、配方法、合同变换法将二次型$f=x_1x_2+x_2x_3+x_3x_1$化为标准型。

3. 化二次型$f=\sum\limits_{i=1}^n x_i^2+\sum\limits_{i<j} x_ix_j$为标准型。

4. $f=(t+1)x_1^2-2x_1x_2+(t+2)x_2^2-2x_2x_3+(t+1)x_3^2$,求$t$范围使$f$为正定二次型。

\subsection{证明}
1. 若$a,b,c$两两正交,求证:$\vert a\vert^2+\vert b\vert^2+\vert c\vert^2=\vert a+b+c\vert^2$。

2. 证明下面各条等价:

(1)$\mathscr{U}$是正交变换 \quad(2)$\mathscr{U}$保范数 \quad(3)$\mathscr{U}$保标准正交基 \quad(4)$\mathscr{U}$在任一标准正交基下对应一个正交方阵

3. 设$V$是$n$维欧氏空间,$v_0\in V, v_0\neq\theta$,求证:$V_0:=\lbrace v\vert\langle v,v_0\rangle=0\rbrace$是$V$的子空间且$dim(V_0)=n-1$。

4. 设$z_1, z_2\cdots z_n$是$n$维欧氏空间$V$的一组向量,求证:$z_1, z_2\cdots z_n$线性无关当且仅当$\vert A\vert\neq 0$,其中$A=(\langle z_i,z_j\rangle)_{n\times n}$。

5. 设$\eta\in V$是一单位向量,定义线性变换$\mathscr{U}:V\rightarrow V$,$v\rightarrow\mathscr{U}(v)=v-2\langle\eta,v\rangle\eta$,求证:(1)$\mathscr{U}$是正交变换。(2)$\mathscr{U}$在任一组标准正交基下的矩阵$U$有$\vert U\vert=-1$。

6. 设$\emph{\textbf{a}}$是单位列向量,记$A=I_n-2\emph{\textbf{a}}\emph{\textbf{a}}^T$,求证:$A$是正交对称阵。

7. 设$A,B$为同阶正交阵且$\vert A\vert=-\vert B\vert$,求证:$\vert A+B\vert=0$。

8. 设$A,B$为$n$阶对称阵且$A$正定,求证:存在非退化阵$P$,使得$P^TAP$和$P^TBP$均为对角阵。

\subsection{投影问题}
1. 设$W$是$V$的线性子空间,对$v\in V$,记$v_W$为$v$在$W$上的正交投影,求证:对$\forall w\in W, \vert v-v_W\vert\leqslant\vert v-w\vert$\quad ($v_W$是$v$在$W$上的最佳逼近)。

反之,若$v_W\in W$满足上述性质,求证:$v-v_W\perp W$。

2. 设$P$是$n$阶投影矩阵,即$P$满足$P^T=P$和$P^2=P$。求证:$P\sim
\begin{pmatrix}
I_r&0\\
0&0
\end{pmatrix}$。
进一步证明:$\exists U\in \mathscr{R}^{n\times r}$,使得$P=UU^T, U^TU=I_r$。

\end{document}
