\documentclass[UTF8,a4paper,10pt,twocolumn]{ctexart}
\usepackage{amsmath,amsfonts,amssymb}
\usepackage{fancyhdr}

\pagestyle{fancy}
\rhead{}
\cfoot{\thepage}
\renewcommand{\headrulewidth}{0pt}

\begin{document}
\title{{\Huge\heiti{综述:稀土金属与铱配合物及其光磁性质}}}
\author{\textbf{171240013 \space 于紫雯 \space 171240008 \space 霍然 \space 171240517 \space 刘宇杰}}
\date{}
\maketitle

\section{铱及磷光}
在元素化学当中,金属铱常与钴、铑分为一类,并且以钴作为其代表元素,主要介绍其性质与化合物、配合物等构成。铱在1803年被发现,并且是在粗铂溶于王水之后剩下的残渣中找到的。S.Tenant在发现铱元素的时候,以希腊女神彩虹命名它,因为它的化合物颜色多种多样。铱是所有元素中最为致密的,具有面心立方结构,是过渡系列中首次出现这种结构的元素。因为根据能带理论计算,当外层d轨道接近充满时面心立方结构要比体心立方或六方密堆积更稳定。

化学家根据余辉——激发停止后晶体发光消失的时间——的长短将晶体的发光分成两类:荧光和磷光。若受激发分子的电子在激发态发生自旋反转,当其所处单重态的较低振动能级与激发三重态的较高能级重叠时,即发生系间跳跃,到达激发三重态,经过振动驰豫到达最低振动能级,以辐射形式发射光子跃迁到基态的任一振动能级上,称此时发射的光子为磷光。

经大量文献报道,铱配合物可以发出多种颜色的磷光。一种方法是以二苯基吡啶(ppy)和$2,2'-$联吡啶$-4,4'-$二羧酸为配体,与水合三氯化铱$(IrCl_3$·$3H_2O)$配位,合成一种具有橙红色磷光发射的金属铱配合物,且具有金属-配体电荷转移吸收和磷光发射的特性。另一种绿色磷光铱配合物基于$Ir(ppy)_2acac$,合成了具有苄醚结构的$2-$苯基吡啶铱(Ⅲ)配合物$[Ir(ppy- OCH_2Ph)_2acac]$,并可以该配合物为磷光材料,制备电致发光器件。此外,高效蓝色磷光铱配合物$[Ir(ppy)_2:(POP)]PF_6$也由二($2-$二苯基磷基)苯基醚(POP)为辅助配体、$2-$苯基吡啶(ppy)为第一配体合成。特别的是,此配合物的发射光谱不仅有主要的磷光,还有配体跃迁发射的荧光。 

有机电致发光器件(OLED)的主体组成需要使用发光材料,包括有机荧光发光材料、有机磷光客体材料和主体材料。对于磷光红光客体材料,起初以铂为中心合成了红色磷光客体染料(PtOEP)。但其三重态寿命过长(80µs),电流密度高时易产生三重态激子间淬灭(TTA),致其发光效率难以提升。后来研发出以铱为金属核的红光磷光材料$(Btp_2Ir(acac))$ ,其三重态周期较短(4µs),最大外量子效率约$7.0 \%$ ,高电流密度下效率衰减较缓慢。此外,三(2-苯基吡啶)合铱$(Ir(ppy)_3)$也是广泛利用的绿光磷光材料之一。由此可见,在磷光发光材料当中,铱起着举足轻重的作用,由于其良好的最大外量子效率和最大功率效率,以及在高电流密度下的弱衰减性,成为广大科研工作者热衷的研究对象。有文献提及,根据铱化合物的光激发现象,可以在铱化合物中引入苯并恶唑基-苯氧基(bop),一者对$Ir(ppy)_3$用bop基团取代一个主配体2-苯基吡啶,制得黄绿光铱化合物$Ir(ppy)_2bop$,二者合成深红光铱化合物 $Ir(piq)_2bop$,获得新型铱化合物及其良好的光学性质。

铱磷光配合物具有效率高、寿命长、可见光激发和可通过改变配体结构调节发光等优点。基于这些优点,其正被广泛应用于传感和成像领域。有文献指出,一种阴离子型铱磷光配合物$TBA[Ir(dfppy)_2(NCS)_2]$,可被用于汞离子检测和细胞染色标记,另一种含有铱磷光配合物和稀土大环多胺多酸稀土配合物双功能单元的异核金属配合物(Ir-Gd)可作为活体磁共振成像造影剂,以期在医学领域发挥作用。

由此可见,铱化合物在发光材料中应用广泛。如今我们希望合成具有发光性质的磁性材料,而铱在发光方面具有较大的改造潜力和应用空间,将作为发光主体材料。下文将介绍的稀土元素具有良好的磁性和可被敏化发光的性质,将之与铱结合,并研究如何使发光和磁性并非两个独立存在而是相互促进、共同拥有的性质,是我们的目标。

\section{稀土元素}
稀土元素一般指ⅢB族的钪、钇和镧到镥的镧系元素。其有机配体配合物所发荧光兼稀土离子发光强度大、颜色纯正和有机化合物激发能量低、荧光效率高之优点。与铱配合物所发磷光不同,稀土配合物之荧光主要缘于稀土离子外层为全充满电子$5s^25p^6$之结构,次外层含未充满的$4f$,会产生$f \rightarrow f^*$吸光跃迁,亦会产生产$f^* \rightarrow f$发光跃迁。然4f层内跃迁宇称禁阻,谱线较窄,吸光发光均较弱,但也因此颜色鲜艳纯正。上世纪60年代Whan与Crosby等人提出了从有机配体到$Ln^{3+}$离子的能量传递机理,即有机配体将吸收光子的能量通过分子内能量转移方式向$Ln^{3+}$离子传递以增强其发光强度,故可通过改变$Ln^{3+}$离子的配位环境来提高其发光效率。此种配体分子对中心稀土离子的敏化效应也被称为“天线”效应。然而配体分子的能级基本决定了其能否有效敏化稀土离子,其三重态能级一般需比$Ln^{3+}$离子的最低激发态高$2000-5000cm^{-1}$。但使用传统有机配体敏化稀土离子,发光弱且寿命短,为追求更高的发光性能,需寻找可见光范围内吸收较强且能级适合敏化发光的配体。据文献报道,2012年意大利Roberta Sessoli课题组利用大环配体DOTA(1,4,7,10-四氮杂-N,N$'$,N$''$,N$'''$-四乙酸)合成得单分子磁体$[Na{Dy(DOTA)(H_2O)}]_4H_2O$,其在365nm激发波长下表现出$Dy^{3+}$特征荧光谱,将光磁性质联系了起来。 

稀土离子敏化发光特性在生命科学和生物化学中有着较为广泛的应用。生物体内主要的钙、镁离子因其为闭壳层电子结构,缺乏相应的光磁性质。然其某些物化性质与具开壳层$4f$结构的稀土离子相近——同属硬酸、离子半径相近、与配体作用以静电力为主、成键方向选择性较差。有文献报道,可将钙离子等替换为稀土离子,通过监测稀土探针的光谱变化,得到与钙离子等有关的生物大分子结构的信息,并可作核酸探针。另有文献报道,稀土卟啉配合物在生命科学中也有良好的应用。卟啉镱(Ⅲ)配合物因其稳定性良好,近红外发光量子产率高,被广泛应用开发多种近红外发光材料。一种卟啉镱(Ⅲ)-四烷基吡啶卟啉二联体Yb.1可用于肿瘤光动力治疗和成像,另一种水溶性卟啉镱(Ⅲ)探针Yb-2可用于线粒体靶向成像且没有细胞光毒性,再一种meso-四(4-(N-咔唑)丁烷氧苯基)卟啉及其稀土(Sm,Eu,Tb,Dy)配合物具有良好的光电性质。由此可见,若是合理开发稀土元素,制造新型材料,可以在诸多领域发挥效用。

过渡金属元素由于其电子层结构的特性,是可见光区强吸收发色团的主要来源。由于稀土元素发光效能较弱,故可考虑由过渡金属元素敏化稀土发光,以提升其发光效率。我们所关注的过渡金属铱,据文献报道,具有良好的敏化其他金属离子发光、光催化等应用,故可设计合理的结构,利用金属铱敏化稀土离子发光,提高发光效能。

\section{相关进展}
为了合成具有发光性质的磁性材料,有几种不同的途径。一种是发光和磁性为两个独立的部分,并通过复合方法将发光中心与磁性中心复合,以满足在化学和生命科学中的应用需求。据文献报道,较好的此种复合材料是以$Fe_3O_4$纳米粒子为磁性核,镧系稀土离子掺杂化合物为荧光壳合成的磁性荧光纳米复合材料。另一种是利用稀土磁性和其他原子——尤以过渡金属——的敏化作用,形成配合物等以达到稀土元素与过渡金属协同发光,发光类型多样并具有磁性的目的,提高材料的利用效率。此种正是我们所希望研究和合成的。

关于稀土元素的磁性,在常温下其均为顺磁物质,但随着温度降低,它们会发生由顺磁性到铁磁性或反铁磁性的有序变化。其磁性主要与未充满的$4f$壳层有关。当今磁性材料主要的研究方面有分子基磁性材料和单分子磁体等,以含有未成对电子的分子及其分子聚集体的磁性质为研究对象的分子磁学发展迅速、成果颇多。分子基磁性材料合成的主要方法是利用合适的高自旋载体,通过有桥联作用的有机配体形成分子内强相互作用和分子单元间弱相互作用结合而成的超分子结构。其研究热点主要集中在有机自由基-金属配合物和金属配位化合物,其主要进展在居里温度较高的磁性材料,据报道现已合成一些居里点接近室温的分子磁体,根据对顺、反-三唑铁(Ⅱ)衍生物($Tc\sim 250K$)的研究表明,配体的顺反异构影响很大。与之相区别的是单分子磁体——宏观上表现出可测的慢磁弛豫行为的单个分子。其主要包括过渡金属单分子磁体和稀土单分子磁体。其中稀土单分子磁体发展迅速、数量众多,且以$[(Cp^{ttt})_2Dy][BPh_4](E=1837K, T_B=60K)$为代表,研究稀土-过渡(异金属)相互作用,通过引入过渡金属离子来增强分子内的磁耦合,提高单分子磁体磁各向异性,抑制体系的量子隧穿行为。

铱的可见光区敏化且发光寿命较长的特点使其备受关注。在Ir(Ⅲ)-Ln(Ⅲ)异核配合物发光性能的方面,2004年De Cola等合成了基于三唑吡啶的形式为$Ir_2Eu$的三核配合物,其因混合了Ir(Ⅲ)的蓝光和Eu(Ⅲ)的红光而发射白光,但Ir(Ⅲ)到Eu(Ⅲ)的传能效率仅为38 \%。北京大学光电功能材料及应用课题组、Ward课题组对此问题研究总结得出,只有当桥联配体的T$_1$比Ir(Ⅲ)的$^3MLCT$能级低且形成共轭体系时,才能实现Ir(Ⅲ)中心到Eu(Ⅲ)中心的有效传能。再者,发射红外光的Ir(Ⅲ)-Yb(Ⅲ)、Nd(Ⅲ)、Er(Ⅲ)等异核配合物也被先后合成。2014年Lina等人首次以铱配合物与稀土离子为单体合成了四种Ir-Ln(Yb,Nd,Er)配位聚合物并且实现$d \rightarrow f$的有效传能。由于含铱单元可在可见光区被激发,实际应用时紫外光照射产生的损伤将得以避免。

如今我们希望研究的金属铱与稀土元素的配合物,并合成具有光学性质的磁性材料,也有较多的进展。2015年,Jana等采用一种刚性共轭的桥联配体将发射强磷光的Ir(Ⅲ)单元与水稳性Gd(Ⅲ)单元相连,得到了发光寿命长、弛豫时间长的光磁材料,在光磁共振双重成像中极具应用潜力。2016年其又报道了可用于氧气检测的双核(Ir·Ln)与三核(Ir·Ln$_2$)配合物,具有较高的应用价值。另一种由铱配合物做配体的稀土元素配合物$Ir(ppy)_2(phen5f)$具有环金属化配体与四齿辅助配体,可被敏化近红外发光。另一种铱配合物$Ir(pdt)_2(phen5f)$可用于合成具有Ir向Yb的能量传送及近红外发光特性的双金属配合物$[(pdt)_2Ir(\mu-phen5f)YbCl_2\cdot 2CH_3CH_2OH\cdot H_2O]Cl$。也可使用蓝色磷光铱配合物作配体与Tb配位并将其敏化。此外,基于d至f轨道能量迁移发光的Ir(Ⅲ)/Eu(Ⅲ)复合物可通过平衡成分使铱发出的蓝色光与铕发出的红色光混合成白色光,其能量转移机制被科研工作者广泛研究,一种可能的方案是以萘基化合物作为空间继承和能量转递的信使。铱和铕的二元复合物也可被用于细胞成像及其他生命科学领域。2014年郑丽敏课题组报道的第一例铱-镝膦酸盐$[DyIr_6(ppy)_{12}(bpp)_2(bppH)_4](CF_3SO_3)\cdot 8H_2O$具有来源于Ir的光致发光和来源于Dy的场致慢磁化弛豫的双重功能,良好地将光学性质和磁学性质结合,具有广泛的应用前景。我们希望设计的,正是将光学性质和磁学性质结合起来的配合物,以发掘元素潜力、发挥更大效能。

\section{配位聚合}

配位聚合物主要由过渡金属与有机配体组装而成,具有结构多样、特殊光电效应、可应用于诸多过渡金属等特点,在非线性光学材料、磁性材料、 超导材料及非对称催化等方面应用广泛。按结构其可被分为三大类:一维链状聚合物、二维网状聚合物和三维网状聚合物。其配体主要由含氮杂环类、含CN类以及混合桥联配体等。其显著特点为有机配体需选择有对称结构的化合物且配位原子主要为N、O。此类配位聚合物相较于一般配合物有显著优点,既保持了配合物的无机有机性质,又具有聚合物良好的物理化学性能,是一个研究的热点。据文献报道,以均苯四甲酸根为桥联配体合成了二维配位聚合物$[Fe(µ_4-bta)_{0.5}(phen)(OH)]_n$,具有反铁磁作用和良好的热稳定性。其他有关稀土元素的配位聚合物$[Na(THF)_2(\mu_2\lambda^5Cp)(\lambda^5,\lambda^5,\eta^1-Cp_3Yb)(THF)]_n$也由自组装合成,形成一维无限的聚合物链结构。另据文献指出,许多稀土元素配位聚合物具有良好的磁学性质,${Gd(TDA)(Ac)(H_2O)}_n$中相邻的$Gd^{3+}$之间存在弱的反铁磁相互作用而${Dy(TDA)(Ac)(H_2O)}_n$显示出三维框架结构中较少的慢磁弛豫行为,具有研究价值。相信我们亦可在此处做出我们的贡献。

\section{参考资料}
\noindent{}[1] Greenwood N N, Earnshaw A. Chemistry of the Elements[M]. Pergamon Press, 1984.

\noindent{}[2] 李大锐, 宋海生, 朱亚超,等. 橙红色磷光铱(Ⅲ)配合物的合成及表征[J]. 广州化工, 2011, 39(7):46-47.

\noindent{}[3] 李襄宏, 胡玉梅. 一种绿色磷光铱配合物的合成、表征及发光性质[J]. 中南民族大学学报(自然科学版), 2010, 29(3):6-9.

\noindent{}[4] 张丽英, 曲凌波, 鲁俊超,等. 高效蓝色磷光铱配合物的合成、表征与发光性能研究[J]. 人工晶体学报, 2013, 42(8).

\noindent{}[5] 黄韵. 基于新型铱化合物高效有机电致发光器件的研究[D]. 电子科技大学, 2015.

\noindent{}[6] 钱俊杰. 铱磷光配合物的合成及其在生物成像中的应用[D]. 上海师范大学, 2010.

\noindent{}[7] 刘艳红, 朱保华. 稀土配合物的发光研究[J]. 赤峰学院学报(自然版), 2011, 27(2):25-26.

\noindent{}[8] 张卫芳, 刘德文. 稀土配合物发光研究及其在生物化学中的应用[J]. 北京轻工业学院学报, 1999, 17(2):60-64.

\noindent{}[9] 张涛, 郑举敦, 吴云霞. 稀土卟啉近红外发光配合物在生命科学领域中的应用研究进展[J]. 激光生物学报, 2016, 25(3):193-197.

\noindent{}[10] 王彬彬, 李瑶, 单凝,等. Meso-四(4-(N-咔唑)丁烷氧苯基)卟啉稀土金属配合物的合成及光电性质[J]. 中国科学:化学, 2013(11):1497-1504.

\noindent{}[11] 吴拓, 潘桦滟, 罗东,等. 基于不同稀土荧光基质的磁性荧光复合材料的研究进展[J]. 功能材料, 2016, 47(s1):83-88.

\noindent{}[12] 李跃华, 杨志毅. 分子基磁性材料的发展和展望[J]. 大理大学学报, 2011, 10(4):43-47.

\noindent{}[13] Zhu Z, Guo M, Li XL, Tang J. Recent advance on single molecule magnets. Sci Sin Chim, 2018, 48: 790–803.

\noindent{}[14] Chen F F, Bian Z Q, Lou B, et al. Sensitised near-infrared emission from lanthanides using an iridium complex as a ligand in heteronuclear Ir2Ln arrays[J]. Dalton Transactions, 2008, 41(41):5577-5583.

\noindent{}[15] Li D, Chen F F, Bian Z Q, et al. Sensitized near-infrared emission of Yb from an Ir–Yb bimetallic complex[J]. Polyhedron, 2009, 28(5):897-902.

\noindent{}[16] Sykes D, Ward M D. Visible-light sensitisation of Tb(III) luminescence using a blue-emitting Ir(III) complex as energy-donor[J]. Chemical Communications, 2011, 47(8):2279.

\noindent{}[17] Sykes D, Tidmarsh I S, Barbieri A, et al. d-f Energy Transfer in a Series of IrIII/EuIII Dyads: Energy-Transfer Mechanisms and White-Light Emission[J]. Inorganic Chemistry, 2011, 50(22):11323-11339.

\noindent{}[18] Gang Yu, Yadong Xing, Dr. Fangfang Chen,等. Energy-Transfer Mechanisms in Ir III –Eu III, Bimetallic Complexes[J]. Chempluschem, 2013, 78(8):852–859.

\noindent{}[19] Sykes D, Parker S C, Sazanovich I V, et al. d-f energy transfer in Ir(III)/Eu(III) dyads: use of a naphthyl spacer as a spatial and energetic "stepping stone".[J]. Inorganic Chemistry, 2013, 52(18):10500-10511.

\noindent{}[20] Baggaley E, Cao D, Sykes D, et al. Corrigendum: Combined Two‐Photon Excitation and d-f Energy Transfer in a Water‐Soluble IrIII/EuIII Dyad: Two Luminescence Components from One Molecule for Cellular Imaging[J]. Chemistry (Weinheim an Der Bergstrasse, Germany), 2014, 20(29):8898.

\noindent{}[21] Jana A, Crowston B J, Shewring J R, et al. Heteronuclear Ir(III)–Ln(III) Luminescent Complexes: Small-Molecule Probes for Dual Modal Imaging and Oxygen Sensing[J]. Inorganic Chemistry, 2016, 55(11):5623-5633.

\noindent{}[22] FangFang Chen, HuiBo Wei, ZuQiang Bian,等. Sensitized Near-Infrared Emission from IrIII-LnIII (Ln = Nd, Yb, Er) Bimetallic Complexes with a (N∧O)(N∧O) Bridging Ligand[J]. Organometallics, 2014, 33(13):3275-3282.

\noindent{}[23] Zeng D, Ren M, Bao S S, et al. A luminescent heptanuclear DyIr6 complex showing field-induced slow magnetization relaxation.[J]. Chemical Communications, 2014, 50(61):8356-8359.

\noindent{}[24] 徐吉庆, 储德清, 于杰辉,等. 有机多羧酸为桥连配体的过渡金属配位聚合物化学--[Fe(μ4-bta)0.5(phen)(OH)]n的合成与结构表征[J]. 中国科学: 化学, 2003, 33(5):434-440.

\noindent{}[25] 黄丹, 孙宏枚, 刘太奇,等. 自组装形成的稀土金属配位聚合物[Na(THF)2(μ2-η5Cp)(η5,η5,η1-Cp3Yb)(THF)]n的合成及结构[J]. 科学通报, 2005, 50(10):958-960.

\noindent{}[26] 牛淑云, 范洪涛, 金晶,等. 配位聚合物[Co2(C3H4N2)4(C10H2O8)]n的晶体结构、磁性及光-电性能[J]. 科学通报, 2004, 49(15):1499-1502.

\noindent{}[27] Ren J, Liu Y, Chen Z, Xiong G, Zhao B. Structures and magnetic properties of several novel lanthanide coordination polymers based on thiophere-2,5-dicarboxylic acid[J]. Sci China Chem, 2012, 5(6): 1073–1078.

\noindent{}[28] Tong H, Jiang Y, Zhang Q, et al. Enhanced Interfacial Charge Transfer on a Tungsten Trioxide Photoanode with Immobilized Molecular Iridium Catalyst[J]. Chemsuschem, 2017, 10(16):3268–3275.

\noindent{}[29] Xie Y, Shaffer D W, Lewandowska-Andralojc A, et al. Water Oxidation by Ruthenium Complexes Incorporating Multifunctional Bipyridyl Diphosphonate Ligands[J]. Angewandte Chemie International Edition, 2016, 55(28):8067-8071.

\noindent{}[30] Neuthe K, Bittner F, Stiemke F, et al. Phosphonic acid anchored ruthenium complexes for ZnO-based dye-sensitized solar cells[J]. Dyes and Pigments, 2014, 104(104):24-33.

\noindent{}[31] Köhler C, Rentschler E. Functionalized phosphonates as building units for multi-dimensional homo- and heterometallic 3d-4f inorganic-organic hybrid-materials[J]. Dalton Trans, 2016, 45(32):12854.

\noindent{}[32] Martir D R, Zysman-Colman E. Supramolecular iridium(III) assemblies[J]. Coordination Chemistry Reviews, 2018, 364:86-117.

\noindent{}[33] Li X, Wu J, Chen L, et al. Engineering an iridium-containing metal-organic molecular capsule for induced-fit geometrical conversion and dual catalysis.[J]. Chemical Communications, 2016, 52(62):9628-9631.

\noindent{}[34] 禹钢,卞祖强,刘志伟,黄春辉.铱(Ir~Ⅲ)-稀土(Ln~Ⅲ)异核配合物发光性能研究[J].中国科学:化学,2014,44(02):267-276.

\noindent{}[35] Cucinotta G, Perfetti M, Luzon J, et al. Magnetic anisotropy in a dysprosium/DOTA single-molecule magnet: beyond simple magneto-structural correlations.[J]. Angew Chem Int Ed Engl, 2012, 124(7):1638-1642.

\noindent{}[36] Li L, Zhang S, Xu L, et al. Highly sensitized near-infrared luminescence in Ir-Ln heteronuclear coordination polymers via light-harvesting antenna of Ir(III) unit[J]. Journal of Materials Chemistry C, 2014, 2(9):1698-1703.
\newpage
\noindent{}[37] Jana A, Pope S J A, Ward M D. D-f energy transfer in heteronuclear Ir(III)/Ln(III) near-infrared luminescent complexes[J]. Polyhedron, 2017, 127.

\end{document}

